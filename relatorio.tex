\documentclass{article}
\usepackage{graphicx}
\usepackage{indentfirst}
\usepackage{amsmath}
\usepackage{amssymb}
\usepackage{float}


\title{Descrição dos Planos Amostrais}
\author{Gabriel Ligabo Baba}
\date{Maio de 2025}
\DeclareMathOperator{\E}{\mathbb{E}}

\begin{document}

\maketitle

\section*{Planos Amostrais Adotados}

Neste trabalho, serão utilizados dois planos amostrais distintos: a Amostragem Aleatória Simples sem Reposição (AASs) e a Amostragem Estratificada Proporcional (AEpr) via AASs.
A escolha dos planos se dá em função do nível geográfico analisado (estado, região, Brasil) e do cargo político (governador, senador e presidente).

\subsection*{Amostragem Aleatória Simples sem Reposição (AASs)}

A AASs consiste em sortear, de forma aleatória e sem reposição, unidades da população $\mathcal{U} = \{1, 2, ..., N\}$, de modo que cada subconjunto de tamanho $n$ possui a mesma probabilidade de ser escolhido. 

Este plano será utilizado para a seleção de eleitores em pesquisas referentes ao cargo de governador e senador, nos estados do Amazonas (AM) e Roraima (RR). 
A escolha desse plano se justifica pela homogeneidade relativa das populações nesses estados e pela simplicidade operacional da técnica.

\subsubsection*{Estimador da Proporção}

Como o objetivo da pesquisa é estimar a proporção de eleitores favoráveis a um determinado candidato "X", o estimador pontual da proporção populacional é dado por:

\[
\hat{P} = p = \frac{m}{n}
\]

em que:
\begin{itemize}
    \item $m$: número de respondentes que indicaram intenção de voto no candidato "X";
    \item $n$: tamanho da amostra.
\end{itemize}

Este estimador é não viciado, como pode ser demonstrado pela esperança:

\[
\E[\hat{P}] = \E\left[\frac{1}{n}\sum_{i \in \mathbf{s}} Y_i\right] = \frac{1}{n} \sum_{i \in \mathbf{s}} \E[Y_i] = P
\]

\subsubsection*{Variância do Estimador}

A variância da estimativa da proporção, assumindo amostragem sem reposição, é dada por:

\[
Var(\hat{P}) = \left(1 - \frac{n}{N}\right) \cdot \frac{P(1 - P)}{n} = \frac{N - n}{N - 1} \cdot \frac{PQ}{n}
\]

Como o valor real de $P$ é desconhecido, utilizamos a estimativa da variância:

\[
var(\hat{P}) = \left(1 - \frac{n}{N} \right) \cdot \frac{p(1 - p)}{n - 1}
\]

Este plano será aplicado com tamanhos amostrais definidos por:

\[
n_{AM}, n_{RR} \text{ para governador e senador}
\]

Os valores de $n$ serão determinados posteriormente com base na margem de erro de $2\%$ e nível de confiança de $95\%$.

\subsection*{Amostragem Estratificada com Alocação Proporcional (AEpr)}

Será adotado o plano AEpr com base em AASs para as pesquisas relacionadas ao cargo de presidente da república, considerando as análises por região e por país. A estratificação será realizada com base nos estados que compõem a região Norte, mantendo a proporcionalidade de suas populações eleitorais na seleção amostral.

\subsubsection*{Estimador da proporção}
É importante ressaltar que, nesse plano amostral, o tamanho dos estratos é dado por:
\[
n_h = nW_h = n\frac{N_h}{N}
\]
Dito isso, podemos prosseguir para o estimador da proporção:
\[
p_{es} = \sum_{h=1}^H W_h p_h = \sum_{h=1}^H \frac{N_h}{N}p_h
\], onde $p_h$ é o estimador da proporção no estrato h.

\[
p_h = \frac{m_h}{n_h}
\]

em que:
\begin{itemize}
    \item $m$: número de respondentes que indicaram intenção de voto no candidato "X" no estrato h;
    \item $n$: tamanho da amostra no estrato h.
\end{itemize}

\subsection*{Variância da proporção}
A variância da estimativa da proporção, assumindo amostragem proporcional com amostragem aleatória simples sem reposição é dada por:
\[
Var(p_{es}) = Var(\sum_{h = 1}^{H}W_h p_h) = \sum_{h = 1}^{H}W_h ^ 2 Var(p_h)
\], onde:

\[
Var(p_h) = \frac{N-n}{N-1}\frac{PQ}{n}
\]

No geral, o verdadeiro valor de P é desconhecido. Então, utilizamos a estimativa da variância:

\[
var(p_{es}) = var(\sum_{h = 1}^{H}W_h p_h) = \sum_{h = 1}^{H}W_h ^ 2 var(p_h)
\], onde:

\[
var(p_h) = (1-f)\frac{pq}{n-1}
\]


\section*{Cálculos de tamanho de amostra}
Com a teoria descrita na seção anterior, podemos prosseguir com os cálculos de tamanho de amostra para
cada um dos casos de interesse.

\subsection*{Governador e senador}

Fixando um erro de 2 pontos percentuais e uma confiança de 95\% ($\alpha = 0.05$), considerando AASs, temos que:

\[
\begin{array}{l}
P\left( \left| \frac{p - P}{\sqrt{ \left( \frac{N - n}{N - 1} \right) \cdot \frac{PQ}{n} }} \right| \leq z_{\alpha/2} \right) = 0{,}95 \text{ com isso,}\\ 
P\left( \left| p - P \right| \leq B \right)
\end{array}
\]

Onde:
\begin{itemize}
    \item $z_{\alpha/2} = 1.96$
    \item $B = z_{\alpha/2}\sqrt{\frac{N-n}{N-1}\frac{PQ}{n}} = 0.02$
\end{itemize}

Realizando as contas considerando o caso conservativo, isto é $P(1-P) = \frac{1}{4}$, chegamos na seguinte expressão:
\[
n = \frac{N}{4(N-1)(\frac{B}{z_{\alpha/2}}^2) + 1}
\]
\subsubsection*{Primeiro turno}
\[
n_{AM} = \frac{N_{AM}}{4(N_{AM}-1)(\frac{0.02}{1.96}^2) + 1 } = \frac{2110875}{4(2110875 - 1) (\frac{0.02}{1.96}^2) + 1} = 2398.2732 = 2399
\]
Amostra gerada com a seed 1
\[
n_{RR} = \frac{N_{RR}}{4(N_{RR}-1)(\frac{0.02}{1.96}^2) + 1 } = \frac{304319}{4(304319-1)(\frac{0.02}{1.96})^2 + 1} = 2382.2127 = 2383
\]
Amostra gerada com a seed 2
\subsubsection*{Segundo turno}

\[
n_{AM} = \frac{N_{AM}}{4(N_{AM}-1)(\frac{0.02}{1.96}^2) + 1 } = \frac{2065079}{4(2065079 - 1) (\frac{0.02}{1.96}^2) + 1} = 2398.2128 = 2399
\]
Amostra gerada com a seed 3
\subsection*{Presidente}
Para presidente, temos que separar em duas categorias : 
\begin{itemize}
    \item Estratos dentro de uma região do país (norte);
    \item Estratos por estado do país.
\end{itemize}

Além disso, temos que separar o cálculo do tamanho do amostra em duas etapas:
\begin{itemize}
    \item Cálculo do tamanho da amostra considerando AASs para cada um dos casos;
    \item Cálculo do tamanho da amostra por estrato considerando AEpr.
\end{itemize}

\subsubsection*{AASs} 
Nessa etapa, temos a mesma teoria dos cálculos para governador e senador. Fixando um erro de 2 pontos percentuais e uma confiança de 95\% ($\alpha = 0.05$):
\[
\begin{array}{l}
P\left( \left| \frac{p - P}{\sqrt{ \left( \frac{N - n}{N - 1} \right) \cdot \frac{PQ}{n} }} \right| \leq z_{\alpha/2} \right) = 0{,}95 \text{ com isso,}\\ 
P\left( \left| p - P \right| \leq B \right)
\end{array}
\]
Onde:
\begin{itemize}
    \item $z_{\alpha/2} = 1.96$
    \item $B = z_{\alpha/2}\sqrt{\frac{N-n}{N-1}\frac{PQ}{n}} = 0.02$
\end{itemize}

Assim, podemos calcular o tamanho da amostra considerando a estratificação da por região e por país

\textbf{Por região:}

Primeiro turno
\[
n_{NO} = \frac{N_{NO}}{4(N_{NO}-1)(\frac{0.02}{1.96}^2) + 1 } = \frac{9925507}{4(9925507 - 1) (\frac{0.02}{1.96}^2) + 1} = 2400.4196 = 2401
\]
Amostrada gerada com a seed 4
Segundo turno:
\[
n_{NO} = \frac{N_{NO}}{4(N_{NO}-1)(\frac{0.02}{1.96}^2) + 1 } = \frac{9675082}{4(9675082 - 1) (\frac{0.02}{1.96}^2) + 1} = 2400.4045 = 2401
\]
Amostrada gerada com a seed 5

\textbf{Pelo país:}

Primeiro turno
\[
n_{BR} = \frac{N_{BR}}{4(N_{BR}-1)(\frac{0.02}{1.96}^2) + 1 } = \frac{123682372}{4(123682372 - 1) (\frac{0.02}{1.96}^2) + 1} = 2400.9534 = 2401
\]

Amostrada gerada com a seed 6

Segundo turno

\[
n_{BR} = \frac{N_{BR}}{4(N_{BR}-1)(\frac{0.02}{1.96}^2) + 1 } = \frac{124252796}{4(124252796 - 1) (\frac{0.02}{1.96}^2) + 1} = 2400.9536 = 2401
\]

Amostrada gerada com a seed 7

\subsubsection*{AEpr}

Agora, podemos utilizar a alocação proporcional para ambos os casos

\textbf{Por região:}


O cálculo de tamanho da amostra é dado por:

\[
n_{ESTADO} = n_{NORTE}\frac{N_{ESTADO}}{N_{NORTE}}
\]


\begin{table}[H]
\centering
\caption{Tamanho da Amostra Estratificada (AEpr) -- Regi\~{a}o Norte}
\resizebox{\textwidth}{!}{
\begin{tabular}{lcc}
\hline
\textbf{Estado (Norte)} & Primeiro Turno & Segundo Turno \\
\hline
Acre (AC) & 111 & 105 \\
Rond\^onia (RO) & 224 & 229 \\
Amazonas (AM) & 511 & 513 \\
Roraima (RR) & 74 & 72 \\
Amap\'a (AP) & 108 & 100 \\
Par\'a (PA) & 1158 & 1166 \\
Tocantins (TO) & 215 & 216 \\
\hline
\end{tabular}
}
\end{table}


Onde para o primeiro turno, cada uma das amostras foram geradas com as seeds : (8,9,10,11,12,13,14)
Já no segundo turno, cada uma das amostras foram geradas com as seguintes seeds : (15, 16, 17, 18, 19, 20, 21)

\textbf{Por país}

O cálculo de tamanho da amostra é dado por:
\[
n_{ESTADO} = n_{BRASIL}\frac{N_{ESTADO}}{N_{BRASIL}}
\]



\begin{table}[H]
\centering
\caption{Tamanho da Amostra Estratificada (AEpr) -- Total por Pa\'{i}s}
\resizebox{\textwidth}{!}{
\begin{tabular}{lcc}
\hline
\textbf{Estado} & Primeiro Turno & Segundo Turno \\
\hline
Esp\'irito Santo (ES) & 44 & 45 \\
Minas Gerais (MG) & 246 & 249 \\
S\~{a}o Paulo (SP) & 527 & 530 \\
Rio de Janeiro (RJ) & 193 & 193 \\
Paran\'a (PR) & 133 & 134 \\
Rio Grande do Sul (RS) & 133 & 133 \\
Santa Catarina (SC) & 88 & 88 \\
Distrito Federal (DF) & 36 & 36 \\
Goi\'as (GO) & 75 & 75 \\
Mato Grosso do Sul (MS) & 31 & 29 \\
Mato Grosso (MT) & 36 & 36 \\
Alagoas (AL) & 36 & 35 \\
Bahia (BA) & 173 & 174 \\
Sergipe (SE) & 27 & 27 \\
Cear\'a (CE) & 110 & 110 \\
Maranh\~{a}o (MA) & 77 & 75 \\
Piau\'i (PI) & 42 & 41 \\
Para\'iba (PB) & 50 & 50 \\
Pernambuco (PE) & 112 & 113 \\
Rio Grande do Norte (RN) & 41 & 41 \\
Acre (AC) & 8 & 9 \\
Rond\^onia (RO) & 17 & 17 \\
Amazonas (AM) & 42 & 39 \\
Roraima (RR) & 5 & 6 \\
Amap\'a (AP) & 9 & 8 \\
Par\'a (PA) & 92 & 91 \\
Tocantins (TO) & 18 & 17 \\
\hline
\end{tabular}
}
\end{table}



\end{document}
