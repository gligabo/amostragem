\documentclass{article}
\usepackage{graphicx}
\usepackage{indentfirst}
\usepackage{amsmath}
\usepackage{amssymb}

\title{Descrição dos Planos Amostrais}
\author{Gabriel Ligabo Baba}
\date{Maio de 2025}
\DeclareMathOperator{\E}{\mathbb{E}}

\begin{document}

\maketitle

\section*{1. Planos Amostrais Adotados}

Neste trabalho, serão utilizados dois planos amostrais distintos: a Amostragem Aleatória Simples sem Reposição (AASs) e a Amostragem Estratificada Proporcional (AEpr) via AASs.
A escolha dos planos se dá em função do nível geográfico analisado (estado, região, Brasil) e do cargo político (governador, senador e presidente).

\subsection*{Amostragem Aleatória Simples sem Reposição (AASs)}

A AASs consiste em sortear, de forma aleatória e sem reposição, unidades da população $\mathcal{U} = \{1, 2, ..., N\}$, de modo que cada subconjunto de tamanho $n$ possui a mesma probabilidade de ser escolhido. 

Este plano será utilizado para a seleção de eleitores em pesquisas referentes ao cargo de governador e senador, nos estados do Amazonas (AM) e Roraima (RR). 
A escolha desse plano se justifica pela homogeneidade relativa das populações nesses estados e pela simplicidade operacional da técnica.

\subsubsection*{Estimador da Proporção}

Como o objetivo da pesquisa é estimar a proporção de eleitores favoráveis a um determinado candidato "X", o estimador pontual da proporção populacional é dado por:

\[
\hat{P} = \frac{m}{n}
\]

em que:
\begin{itemize}
    \item $m$: número de respondentes que indicaram intenção de voto no candidato "X";
    \item $n$: tamanho da amostra.
\end{itemize}

Este estimador é não viciado, como pode ser demonstrado pela esperança:

\[
\E[\hat{P}] = \E\left[\frac{1}{n}\sum_{i \in \mathbf{s}} Y_i\right] = \frac{1}{n} \sum_{i \in \mathbf{s}} \E[Y_i] = P
\]

\subsubsection*{Variância do Estimador}

A variância da estimativa da proporção, assumindo amostragem sem reposição, é dada por:

\[
Var(\hat{P}) = \left(1 - \frac{n}{N}\right) \cdot \frac{P(1 - P)}{n} = \frac{N - n}{N - 1} \cdot \frac{PQ}{n}
\]

Como o valor real de $P$ é desconhecido, utilizamos a estimativa da variância:

\[
\widehat{Var}(\hat{P}) = \left(1 - \frac{n}{N} \right) \cdot \frac{\hat{P}(1 - \hat{P})}{n - 1}
\]

Este plano será aplicado com tamanhos amostrais definidos por:

\[
n_{AM}, n_{RR} \text{ para governador e senador}
\]

Os valores de $n$ serão determinados posteriormente com base na margem de erro de $2\%$ e nível de confiança de $95\%$.

\subsection*{Amostragem Estratificada com Proporcionalidade (AEpr)}

Será adotado o plano AEpr com base em AASs para as pesquisas relacionadas ao cargo de presidente da república, considerando as análises por região e por país. A estratificação será realizada com base nos estados que compõem a região Norte, mantendo a proporcionalidade de suas populações eleitorais na seleção amostral.

\subsubsection*{Estimador }

\section{Cálculos de tamanho de amostra}

\end{document}
